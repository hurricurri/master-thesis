\chapter{Introduction}
\lhead{\emph{Introduction}}
The term ``self-testing" was coined by Mayers and Yao in 2004 \cite{Mayers2003}. Despite its adolescent age, the field is already at forefront of quantum cryptography and the pillar on which countless protocols rest CITE. The aim of self-testing is to acquire knowledge about the quantum properties of an adversarial input-output device from a minimal set of assumptions, by only interacting with the device classically, i.e.\ passing an input bitstring and receiving an output bitstring for each input-output cycle. The vast majority of self-testing protocols rely on Bell non-locality. As such, they assume that the device is a multipartite system that is split into space-like separated subsystems that are non-communicating for the duration of one input-output cycle. The key observation giving rise to the notion of self-testing is that there exist Bell non-local correlations that are compatible only with an essentially unique quantum model, comprised of quantum state and measurement operators. Bell non-locality further certifies that the input-output correlations were not simulated by a hackable classical device. While
many technical tools have been developed to bound compatible quantum models for Bell-type self-testing scenarios, the task of certifying quantum properties of unipartite quantum system remains largely underexplored. A promising undertaking is to identify sets of assumptions that facilitate self-testing of a unipartite system, without using entanglement and Bell non-locality as relevant resources. Apart from providing insight into the geometry of quantum correlations for more general scenarios, the potential usefulness of such protocols lies in the fact that the no-communication assumption underlying conventional self-testing approaches is hard to implement in practice. Additionally, entanglement is often a very costly resource. The natural starting point we take on in this thesis is to study the role of Kochen-Specker contextuality as a resource for self-testing, as it can be seen as an extension of Bell non-locality to more general, unipartite scenarios. 

This thesis discusses and recasts previous results \cite{Bharti2019} that identify Kochen-Specker contextuality as a resource for self-testing to a general prepare-and-measure scenario. In particular, we propose a protocol based on Spekkens contextuality, discuss its features, and identify conditions that allow us to relax unphysical restrictions on measurement operators, imposed in \cite{Bharti2019}. Along the way, we characterize the resources in terms of memory needed by a pre-programmed classical device to simulate the optimal correlations to the protocol in \cite{Bharti2019}, and present an improved proof of robustness to noise.

The thesis is organized as follows:
In Section \ref{sec:kscontextuality}, we introduce the notion of Kochen-Specker contextuality, which aims to capture the way in which quantum mechanics deviates from our classical intuition, even for unipartite systems. We will derive a class of Kochen-Soecker non-contextuality inequalities in Section \ref{sec:kcbs} that delimit classical from quantum correlations, analogously to Bell inequalities in the multipartite setting. In Section \ref{sec:csw}, we introduce graph theoretic tools, proposed in \cite{Cabello2014}, that allow for a rigorous analysis of these Kochen-Specker non-contextuality inequalities. In particular, these tools enable us to identify and study optimal classical behaviours that saturate the non-contextual bounds, as well as optimal quantum models that produce maximal violations of the inequalities, relevant to self-testing. Section \ref{sec:self-testing} introduces the concept of self-testing in the context of multipartite Bell scenarios, which allows for powerful inferences about the quantum properties of a correlation experiment from minimal assumptions. Additionally, Section \ref{sec:self-testing} points out in what ways this concept can be modified to accommodate unipartite systems with no space-like separation. Equipped with the graph-theoretic tools presented in Section \ref{sec:csw}, we will demonstrate that the Kochen-Specker non-contextuality inequalities in \ref{sec:kcbs} facilitate noise-robust self-testing, as was proved in \cite{Bharti2019}, and improve the robustness to noise. Section \ref{sec:kcbs} also identifies the assumptions that go into this self-testing protocol. In particular, we find that one has to bound the information carrying capacity of the device to exclude the possibility of a pre-programmed classical computer generating the correlations. In an attempt to relax some of the unphysical constraints, such as the restriction to sharp measurements, we shift to a general prepare-and-measure scenario. Spekkens revised notion of non-contextuality, which will be introduced in Section \ref{sec:spekkcont}, is formulated within this general framework. Spekkens operational notion of contextuality will provide us with more suitable and robust tests of non-classicality.
Finally, in Section \ref{sec:protocols}, we submit a self-testing protocol based on Spekkens contextuality and discuss its merits and drawbacks, compared to the protocol in Section \ref{sec:contselftesting}.