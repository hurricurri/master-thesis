\chapter{Introduction}
\lhead{\emph{Introduction}}
This thesis recasts previous results that identify contextuality as a resource to certify the behaviour of a unipartite adversarial quantum device. their results in Spekkens operational framework for contextuality,


our recent work on operational notions of contextuality, with results on self-testing from contextuality, to obtain operational tests.
There are some results out there on using contextuality to certify the behaviour of an adversarial quantum computer [2], a task called self-testing [1]. However, the assumptions they rely on are not operational (and unrealistic). Our goal is to first recast their results in Spekkens operational framework for contextuality, and then to use our results [3] to relax the assumptions and make them more realistic.

In Section \ref{sec:kscontextuality}, we will introduce the notion of KS, which aims to capture the way in which quantum mechanics deviates from our classical intuition, even for unipartite systems. We will derive a class of KS non-contextuality inequalities in Section \ref{sec:kcbs} that delimit classical from quantum correlations, analogously to Bell inequalities for multipartite systems. In Section \ref{sec:csw}, we introduce graph theoretic tools, proposed in \cite{Cabello2014}, that allow for a rigorous analysis of these KS non-contextuality inequalities. In particular, these tools enable us to identify and study optimal classical behaviours that saturate the non-contextual bound, as well as optimal quantum models that produce maximal violations of the inequality, relevant to self-testing. Section \ref{sec:self-testing} introduces the concept of self-testing in the context of multipartite Bell scenarios, which allows for powerful inferences about the quantum properties of a correlation experiment from minimal assumptions. Additionally, Section \ref{sec:self-testing} points out in what ways this concept can be modified to accommodate unipartite systems with no space-like separation. Equipped with the graph-theoretic tools presented in Section \ref{sec:csw}, we will demonstrate that the KS non-contextuality inequalities in \ref{sec:kcbs}facilitate noise-robust self-testing, as was proved in \cite{Bharti2019}, and improve the previous self-testing bounds. Section \ref{sec:kcbs} also identifies the assumptions that go into this self-testing protocol. In an attempt to relax some of the unphysical constraints, such as the restriction to idealized, sharp measurements, imposed by the protocol, Section \ref{sec:spekkcont} introduces a revised notion of non-contextuality due to Spekkens, which will serve as a more suitable test of non-classicality. In particular, as sketched in Section \ref{sec:spekkensineq}, the previous KS non-contextuality inequalities can be lifted to Spekkens non-contextuality inequalities. Finally, in Section \ref{sec:protocols}, we submit two self-testing protocols based on Spekkens contextuality and discuss their merits and drawbacks, compared to the protocol in Section \ref{sec:contselftesting}.