\chapter{Proof of Theorem REF}
\lhead{\emph{Appendix A}}
\label{sec:appendix}

\section{Notation and preliminary considerations}
Denote the density operator corresponding to the preparation $P_{m_k\vert M_i}$ by 
\begin{equation}
\label{eqn:densityop}
\rho_{m_k\vert M_i} = \sum_m p_m^{m_k\vert M_i} \vert \Psi_m^{m_k\vert M_i}\rangle\langle \Psi_m^{m_k\vert M_i} \vert\thinspace ,
\end{equation} where $\vert \Psi_m^{m_k\vert M_i} \rangle \in \mathbb{C}^k$, and the positive semi-definite operator corresponding to the measurement event $m_k\vert M_i$ by
\begin{equation}
\label{eqn:mntop}
F_{m_k\vert M_i}=\sum_l \lambda_l^{m_k\vert M_i} \vert \alpha_l^{m_k\vert M_i}\rangle \langle \alpha_l^{m_k\vert M_i} \vert\thinspace ,
\end{equation} where $\vert \alpha_l^{m_k\vert M_i} \rangle \in \mathbb{C}^k$.
By \ref{eqn:epsilon},
\begin{align*}
p(m_k\vert M_i, P_{m_k\vert M_i}) & \geq 1-2\epsilon \\
p(m_k\vert M_i, P_{m_{k'}\vert M_i}) & \leq \epsilon\, ,
\end{align*}
for $k'\neq k$.
Therefore,
\begin{equation}
\label{eqn:cond1}
\sum_{l,m}\lambda_l^{m_k\vert M_i}p_m^{m_k\vert M_i}\vert \langle \alpha_l^{m_k\vert M_i} \vert \Psi_m^{m_k\vert M_i} \rangle \vert^2 \geq 1-2\epsilon
\end{equation}
and analogously, for $k'\neq k$,
\begin{equation}
\label{eqn:cond2}
\sum_{l,m}\lambda_l^{m_k\vert M_i}p_m^{m_{k'}\vert M_i}\vert \langle \alpha_l^{m_k\vert M_i} \vert \Psi_m^{m_{k'}\vert M_i} \rangle \vert^2 \leq \epsilon.
\end{equation}

Consider the density operator $\rho_{m_k\vert M_i}$, like in \ref{eqn:densityop}. For eigenvalues $p_m^{m_k\vert M_i}$ of the noisy density operator that are very small, say of the order $\epsilon$, we cannot infer from the statistics $\{p(m_j\thinspace\vert\thinspace M_i, P_{m_k\vert M_i})\}_{j,k}$ alone much about how the accessible measurements act on the corresponding subspace $\mathbb{C}\vert \Psi_m^{m_k\vert M_i}\rangle$, since this subspace is ``insignificant" in terms of statistics. To sidestep this issue, we consider only a ``statistically relevant" subspace of $\mathbb{C}^k$:
Define $\Pi_{\text{relev}}^{(i)}$ as the projector onto the subspace 
\begin{equation}
\label{eqn:relevsubspace}
V_{\text{relev}}^{(i)}=\operatorname{span}\left(\thinspace\bigcup_{k=1}^3\{\vert \Psi_m^{m_k\vert M_i}\rangle\}_{m:p_m^{m_k\vert M_i}\geq\mathcal{X}}\right).
\end{equation}
For now, $\mathcal{X}$ is an arbitrary cutoff, characterizing the minimum magnitude of eigenvalues for which the corresponding eigenvector in \ref{eqn:densityop} spans a statistically relevant one-dimensional subspace. We will later set $\mathcal{X}$ to an appropriate value in terms of $\epsilon$. 

The conditions \ref{eqn:cond1} and \ref{eqn:cond2} become
\begin{equation}
\label{eqn:cond1new}
\sum_{m:p_m^{m_k\vert M_i}\geq \mathcal{X}} \sum_l \,(\thinspace 1- \lambda_l^{m_k\vert M_i}) \vert \langle \alpha_l^{m_k\vert M_i}\vert \Psi_m^{m_k\vert M_i} \rangle \vert^2\thinspace \leq \frac{2\epsilon}{\mathcal{X}}
\end{equation}
and
\begin{equation}
\label{eqn:cond2new}
\sum_{l,m:p_m^{m_{k'}\vert M_i}\geq \mathcal{X}} \lambda_l^{m_k\vert M_i} \vert \langle \alpha_l^{m_k\vert M_i}\vert \Psi_m^{m_{k'}\vert M_i} \rangle \vert^2\leq \frac{\epsilon}{\mathcal{X}},
\end{equation}
for $k'\neq k$.

For notational simplicity, we will write $\sum_{m:p_m^{m_k\vert M_i}\geq \mathcal{X}}$ as $\sum'_m$, whenever it is clear what measurement event $m_k\thinspace\vert\thinspace M_i$ we are referring to, and analogously for two summation indices.

\section{Overlap between statistically relevant states}
\label{sec:boundingoverlap}
We begin by bounding the overlap $\vert \langle \Psi_k^{m_j\vert M_i}\vert \Psi_l^{m_{j'}\vert M_i}\rangle \vert$, for $j'\neq j$ and $\vert \Psi_k^{m_j\vert M_i}\rangle$, $\vert \Psi_l^{m_{j'}\vert M_i}\rangle \in V_{\text{relev}}^{(i)}$. In the ideal case these vectors are perfectly orthogonal. 
Using the fact that $\sum_{\Tilde{j}}F_{m_{\Tilde{j}}\vert M_i}=\mathbb{1}$, we can re-write
\begin{equation*}
\sum_{k,l}{}^{'}\thinspace\vert\langle \Psi_k^{m_j\vert M_i} \vert \Psi_l^{m_{j'}\vert M_i}\rangle \vert = \sum_{k,l}{}^{'}\thinspace \vert \thinspace\sum_{\Tilde{j}} \operatorname{tr}\left(F_{m_{\Tilde{j}}\vert M_i}\thinspace\vert \Psi_k^{m_j\vert M_i}\rangle\langle \Psi_l^{m_{j'}\vert M_i}\vert\thinspace\right) \vert.
\end{equation*}
This is upper-bounded by
\begin{equation*}
\sum_{k,l}{}^{'}\vert\langle \Psi_k^{m_j\vert M_i} \vert \Psi_l^{m_{j'}\vert M_i}\rangle \vert \leq \sum_{\Tilde{j},m}\sum_{k,l}{}^{'}\thinspace\lambda_m^{m_{\Tilde{j}}\vert M_i}\vert \langle \alpha_m^{m_{\Tilde{j}}\vert M_i} \vert \Psi_k^{m_j\vert M_i} \rangle \vert \thinspace \vert \langle \Psi_l^{m_{j'}\vert M_i}\vert \alpha_m^{m_{\Tilde{j}}\vert M_i} \rangle \vert.
\end{equation*}
Note that since $j\neq j'$, $\Tilde{j}\neq j$ or $\Tilde{j}\neq j'$. We now use condition \ref{eqn:cond2new}, which bounds the square of the 2-norm of the vector with entries
\begin{equation*}
\left((\lambda_m^{m_{\Tilde{j}}\vert M_i})^{1/2}\thinspace \vert \langle \alpha_m^{m_{\Tilde{j}}\vert M_i}\vert \Psi_k^{m_j\vert M_i}\rangle \vert\thinspace\right)_{m,\thinspace l:p_l^{m_j\vert M_i}\geq \mathcal{X}}\;,
\end{equation*} 
for $\Tilde{j}\neq j$. Such vector has at most $d^2$ components, where $d$ is the upper bound on the Hilbert space dimension we imposed in Section \ref{sec:certifyquant}. Since $\|\cdot\|_1$ and $\|\cdot\|_2$ are equivalent, with $\|\vec{x}\|_1\leq \sqrt{k}\thinspace\|\vec{x}\|_2$ for $\vec{x}\in\mathbb{C}^k$, we find
\begin{equation*}
\sum_{k,m}{}^{'}\thinspace\lambda_m^{m_{\Tilde{j}}\vert M_i}\vert \langle \alpha_m^{m_{\Tilde{j}}\vert M_i}\vert \Psi_k^{m_j\vert M_i}\rangle \vert \leq d\thinspace \epsilon^{1/2}\mathcal{X}^{-1/2},
\end{equation*}
for $\Tilde{j}\neq j$, and

\begin{equation}
\label{eqn:overlapbound}
\sum_{k,l}{}^{'}\thinspace\vert\langle \Psi_k^{m_j\vert M_i} \vert \Psi_l^{m_{j'}\vert M_i}\rangle \vert \leq 3\sqrt{2}\thinspace d^2 \epsilon^{1/2}\mathcal{X}^{-1/2}.
\end{equation}
We could drop the factor $3$, since for sufficiently small $\epsilon$, the set of all vectors is linearly independent.

\subsubsection{Why can we confine our analysis to the subspace $V_{\text{relev}}^{(i)}$?}
We now discuss in what sense the assumption regarding operational equivalences of preparations, introduced at the beginning of Section \ref{sec:boundingmodels}, justifies confining our analysis to only the subspace $V_{\text{relev}}^{(i)}$. 
For the convex combinations $S_i$, as defined in \ref{eqn:approxequiv}, we find:
\begin{equation*}
\|\sum_j p_j^{(i)}(\rho_{m_j\vert M_i}-\Pi_{\text{relev}}^{(i)}\rho_{m_j\vert M_i}\Pi_{\text{relev}}^{(i)})\|_F \leq \sum_j p_j^{(i)}\|\thinspace\rho_{m_j\vert M_i}-\Pi_{\text{relev}}^{(i)}\rho_{m_j\vert M_i}\Pi_{\text{relev}}^{(i)}\thinspace\|_F \leq d^{1/2}\mathcal{X}.
\end{equation*}
Due to operational equivalence $S_i\sim S_{*}$, we can write:
\begin{equation*}
\|\sum_j q_j(\rho_{j}-\Pi_{\text{relev}}^{(i)}\rho_{j}\Pi_{\text{relev}}^{(i)})\thinspace\|_F \leq d^{1/2}\mathcal{X},
\end{equation*}
where $\rho_{j}$ is the density operator corresponding to the preparation $P_j$.
Therefore,
\begin{equation}
\label{eqn:sumtobound}
q_0\|(\rho_{0}-\Pi_{\text{relev}}^{(i)}\rho_{0}\Pi_{\text{relev}}^{(i)})\|_F \leq d^{1/2}\mathcal{X},
\end{equation}
where we have used Lemma \ref{lem:posopsum}:
\begin{lemma}
\label{lem:posopsum}
Let $A$, $B$ be two positive semi-definite operators. Then
\begin{equation*}
\|A+B\|_F \geq \|A\|_F.
\end{equation*}
\end{lemma}
\begin{proof}
$\|A+B\|_F^2=\operatorname{tr}(A^2+B^2+2AB)\geq \operatorname{tr}(A^2)=\|A\|_F^2$, where we have used that, due to semi positive-definiteness $tr(B^2)$, $tr(AB)$ $\geq 0$. This can be seen by simply inserting an arbitrary spectral decomposition for $A$, $B$, and noting that all eigenvalues are non-negative.
\end{proof}
The operators in \ref{eqn:sumtobound} are indeed positive semi-definite: they are Hermitian and satisfy
\begin{equation*}
\langle \phi \vert \rho_{j}-\Pi_{\text{relev}}^{(i)}\rho_{j}\Pi_{\text{relev}}^{(i)} \vert \phi \rangle \geq 0,
\end{equation*}
for all $\vert \phi\rangle\in\mathbb{C}^k$. This can be seen by expanding an arbitrary $\vert \phi \rangle $ in terms of an ONB with respect to which the projector $\Pi_{\text{relev}}^{(i)}$ is diagonal.
Since $q_0\approx\frac{1}{3}$, we conclude that
\begin{equation}
\label{eqn:justification}
\|(\rho_{0}-\Pi_{\text{relev}}^{(i)}\rho_{0}\Pi_{\text{relev}}^{(i)})\|_F \leq \mathcal{O}(d^{1/2}\mathcal{X}).
\end{equation}
Apart from when certifying quantumness, as detailed in Section \ref{sec:certifyquant}, we only care about single measurements acting on the distinguished preparation $P_0$ for self-testing. Since $\rho_0$ has no statistically relevant spectral components not in $V_{\text{relev}}^{(i)}$, we are justified in proving almost-projectiveness of the measurements $M_i$ only for the subspace \ref{eqn:relevsubspace}. We will use \ref{eqn:justification} in Section \ref{sec:piecing}, to arrive at the self-testing result.